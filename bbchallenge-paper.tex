% !TeX root = bbchallenge-paper.tex

\title{bbchallenge paper}
\author{
        bbchallenge's contributors
}

\documentclass[a4paper,british]{article}

\usepackage{babel}
\usepackage[utf8]{inputenc}
\usepackage[margin=1in]{geometry}
%\usepackage{subfig}
\usepackage[hidelinks]{hyperref}
\usepackage{caption}
\usepackage{floatpag}
\usepackage{subcaption}
\usepackage{tikz}

\usepackage{algorithm}
\usepackage[noend]{algpseudocode}


\usepackage{graphicx}
\usepackage{mathtools}

\usepackage{amsmath,amsfonts,amssymb,amsthm}

\theoremstyle{definition} % don't use italics
\newtheorem{theorem}{Theorem}[section]
\newtheorem{definition}{Definition}[section]
\newtheorem{lemma}{Lemma}[section]
\newtheorem{proposition}{Proposition}[section]
\newtheorem{corollary}{Corollary}[section]
\numberwithin{equation}{section}


\theoremstyle{definition} % emphasize the "Remark N." with italics, not bold
\newtheorem{observation}{Observation}[section]
\newtheorem{example}{Example}[section]
\newtheorem{remark}{Remark}[section]

\usepackage{xassoccnt}
\DeclareCoupledCountersGroup{theorems}
\DeclareCoupledCounters[name=theorems]{theorem,definition,lemma,proposition,corollary,remark,observation,example}

\usepackage{microtype,xspace,wrapfig,multicol} 
\usepackage[textsize=tiny,color=lightgray]{todonotes} 
\usepackage[normalem]{ulem} % sout
\usepackage{stmaryrd}

\newcommand{\ts}[1]{{\color{red}#1}}
\newcommand{\tsi}[1]{\todo[inline]{TS: #1}}
\newcommand{\tsm}[1]{\todo{TS: #1}}
\newcommand{\tss}[2]{{\ts{\sout{#1}}} {\ts{#2}}}
\newcommand{\jb}[1]{{\color{blue}#1}}
\newcommand{\jbi}[1]{\todo[inline]{JB: #1}}
\newcommand{\jbm}[1]{\todo{JB: #1}}
\newcommand{\jbs}[2]{{\jb{\sout{#1}}} {\jb{#2}}}
\newcommand{\tabi}{\hspace{\algorithmicindent}}
\newcommand{\Lim}[1]{\raisebox{0.5ex}{\scalebox{0.8}{$\displaystyle \lim_{#1}\;$}}}
\newcommand{\N}{\mathbb{N}}
\newcommand{\Z}{\mathbb{Z}}

\newcommand{\lhead}[1]{\stackrel{#1}\triangleleft}
\newcommand{\rhead}[1]{\stackrel{#1}\triangleright}

\newcommand{\tm}[1]{\href{https://bbchallenge.org/#1}{\nolinkurl{#1}}}

\usepackage{xcolor}

\definecolor{colorA}{RGB}{255,0,0}
\definecolor{colorB}{RGB}{255,128,0}
\definecolor{colorC}{RGB}{0,0,255}
\definecolor{colorD}{RGB}{0,255,0}
\definecolor{colorE}{RGB}{255,0,255}

\begin{document}
\date{}
\maketitle

\begin{abstract}
    TODO
\end{abstract}


\setcounter{tocdepth}{2}
\tableofcontents

\section{Introduction}
\subsection{Busy beaver functions}
\subsection{The busy beaver scale: weighing open problems in mathematics}
\subsection{Formal verification: Coq}

\section{The road to S(4) and S(2,4)}
\include{sections/decider-0-cyclers}
\include{sections/decider-1-translated-cyclers}
% !TeX root = ../bbchallenge-paper.tex

\newpage
\subsection{n-gram Closed Position Set (CPS)}\label{sec:n-gramCPS}
% !TeX root = ../bbchallenge-paper.tex

\newpage
\subsection{Repeated Word List (RepWL)}\label{sec:RepWL}

\subsection{$S(4)=107$ and $S(2,4)=3,932,964$}

\section{The road to S(5)}
\include{sections/decider-4-FAR}
% !TeX root = ../bbchallenge-paper.tex

\newpage
\subsection{Meet-in-the-middle weighted FAR (MitM WFAR)}\label{sec:finite-automata-reduction}



\subsection{Sporadic 5-state Turing machines}
\subsection{$S(5)=47,176,870$}

\section{Beyond S(5)}
% !TeX root = ../bbchallenge-paper.tex

A law of Busy Beaver study is that those who prove $BB(n) = x$ claim it is impossible to do the same for $BB(n+1)$. Despite this ignominious history, we cautiously posit $BB(6)$ will be extremely hard, if not impossible, to prove. Unlike previous investigators, we base our claim not on the strength of modern computation and the galatic size of large TMs (the current $BB(6)$ champion runs for far, far more steps than there are atoms in the universe), but on the mathematical hardness of the machines that remain.
We call mathematically difficult TMs \textit{cryptids}, and will discuss them later in this section. We know of cryptids in every part of the Busy Beaver frontier $BB(2,5)$, $BB(3,3)$, and $BB(6,2) = BB(6)$. Our canonical example is ``Antihydra," a machine which encodes a problem very similar to the infamous Collatz Conjecture.

\begin{example}
    ``Antihydra'' \\
    ``Antihydra" 1RB1RA\_0LC1LE\_1LD1LC\_1LA0LB\_1LF1RE\_\-\-\-0RA \\
    Let $f(n) = n + \lfloor \frac{n}{2} \rfloor$, or, 
\end{example}
\subsection{$S(6)$ cryptids}
\subsection{$S(3,3)$ cryptids}
\subsection{$S(2,5)$ cryptids}
\subsection{The Beaver Mathematical Olympiad (BMO)}

\section{Conclusion: the busy beaver frontier}

\bibliographystyle{abbrv}
\newpage
\bibliography{bbchallenge-paper}

\end{document}
